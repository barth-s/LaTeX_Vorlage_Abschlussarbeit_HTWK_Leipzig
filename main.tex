%%%%%%%%%%%%%%%%%%%%%%%%%%%%%%%%%%
% HTWK Leipzig LaTeX Vorlage - Abschlussarbeit
%
% Die Vorlage orientiert sich an der "Richtlinie für die Anfertigung von wissenschaftlichen Arbeiten" der Fakultät Ingenieurwissenschaften.
% Es besteht kein Anspruch auf Richtigkeit.
% 
% Das Dokument basiert teilweise auf den Vorlagen von :
% Markus Voerkel
% https://de.overleaf.com/latex/templates/thesis-template-for-hochschule-fur-technik-wirtschaft-und-kultur-leipzig/nqpftcjmmtts
%
% Linda Vogel & Jon Arnt Kårstad
% https://www.overleaf.com/latex/templates/template-projekt-htwk/mphqwccfvvwy
%
% Benutzung des aktualisierten "acronym" Paketes von Tobias Oetiker als "acronym_update.sty" um Kompatibilität sicherzustellen
% Lizenz des Paketes: "The LaTeX Project Public License 1.3"
% https://ctan.org/pkg/acronym
%
%%%%%%%%%%%%%%%%%%%%%%%%%%%%%%%%%%
%
% Veröffentlicht unter CC BY-SA 4.0 - Lizenz
%
%%%%%%%%%%%%%%%%%%%%%%%%%%%%%%%%%%
%
% Hinweis: Ordner für Abbildungen, PDFs etc. müssen eigenhändig erstellt werden da die Vorlage keine leeren Ordner enthalten darf !
%
%%%%%%%%%%%%%%%%%%%%%%%%%%%%%%%%%%

%---------------------------------------
%	Dokumentenklasse
%---------------------------------------

% Dokumentenklasse %
\documentclass[
ngerman,
toc=flat,
toc=chapterentrywithdots,
captions=tableabove,
listof=entryprefix,
listof=leveldown,
fontsize=12pt,
numbers=noenddot]
{scrreprt}

%---------------------------------------
%	Laden von Paketen
%---------------------------------------

\usepackage{babel}
\usepackage{lmodern}
\usepackage[T1]{fontenc}
\usepackage{float}
\usepackage{ragged2e}

% Geoemetry %
\usepackage{geometry}
\geometry{
	top = 25mm,
	headsep = 7mm,
	left = 28mm,
	right = 20mm, 
	bottom = 25mm,
	}

\usepackage[letterspace=150]{microtype}
\usepackage[onehalfspacing]{setspace}


% Caption %

% Hinweis: Zeilenumbrüche bei mehrzeiligen Bildunterschriften manuell mit \\ setzen 

\usepackage[labelfont={bf,sf},font={bf}, labelsep=space, singlelinecheck=off]{caption} 

\captionsetup[figure]{justification=Centering}
\captionsetup[table]{justification=raggedright}


% BibLaTeX %
\usepackage[
    backend=biber,
    style=numeric-comp,
    sorting=none,
    defernumbers=true
]{biblatex}
\addbibresource{Literatur.bib}

\usepackage{csquotes}
\usepackage{amsmath}
\usepackage{graphicx}

% Formel- und Abkürzungsverzeichnis %
\usepackage[noforwardlinks,printonlyused]{Konfigurationsdateien/acronym_update}

% Konfiguration Abkürzungsverzeichnis %
\newcommand{\abb}[3]{\acro{#1}[{#2}]{#3}}

% Konfiguration Formelverzeichnis %
\newcommand{\form}[4]{\acro{#1}[\ensuremath{#2}]{#4{\acroextra{\hspace{5em}[#3]}}}}
\renewcommand*{\aclabelfont}[1]{\textbf{\boldmath\acsfont{#1}}}
\newcommand{\acls}[1]{\acl{#1} \acs{#1}}

% Hyperref %
\usepackage{hyperref}
\hypersetup{ 
    colorlinks=true,
    linkcolor=black,
    urlcolor=blue,
    citecolor=black}
 
% Einfügen von PDF-Dateien %
\usepackage{pdfpages}

% Einfacher Umgang mit Einheiten %
\usepackage{siunitx}
\sisetup{
    locale=DE,
    per-mode=fraction,
}

% Bessere Kompatibilität der Dokumentenklasse mit div. Paketen %
\usepackage{scrhack}

%---------------------------------------
%	Weitere Konfigurationen
%---------------------------------------

%Kopfzeile mit Seitenzahl
\usepackage[autooneside=false]{scrlayer-scrpage}
\clearpairofpagestyles
\setkomafont{pageheadfoot}{\footnotesize}
\chead{{--~}\pagemark{~--}}


%Konfiguration Verzeichnisse
\BeforeStartingTOC[toc]{\singlespacing} 
\BeforeStartingTOC[lot]{\renewcommand\autodot{:}}
\BeforeStartingTOC[lof]{\renewcommand\autodot{:}}


%Anpassung der Kapitelüberschrift
\renewcommand*{\chapterpagestyle}{scrheadings}
\RedeclareSectionCommand[%
beforeskip=0pt,
afterskip=16pt,
afterindent = false,
font=\LARGE]{chapter}

%Verhindere Zeilenumbruch innerhalb \cite[Seite]{quelle}
\renewcommand*{\prenotedelim}{\addnbspace}
\renewcommand*{\postnotedelim}{\addcomma\addnbspace}
\renewcommand*{\multicitedelim}{\addcomma\addnbspace}
\renewcommand*{\extpostnotedelim}{\addnbspace}
\renewcommand*{\volcitedelim}{\addcomma\addnbspace}


%%%%%%%%%%%%%%%%%%%%%%%%%%%%%%%%%%%%%%%%%%%%%%%%%
%
%       Bitte hier die eigenen Daten eingeben
%       (für Generierung der Titelseite etc.)
%
%%%%%%%%%%%%%%%%%%%%%%%%%%%%%%%%%%%%%%%%%%%%%%%%%

\newcommand{\autor}{Max Mustermann} % Vorname Name
\newcommand{\mnr}{12345} % Matrikelnummer, bspw.: 12345

\newcommand{\betreuerI}{Prof. Dr.-Ing. Moritz Musterprof} % Name 1. Betreuer bzw. betreuender Prof

\newcommand{\betreuerII}{Dipl.-Ing. Manuela Musteringenieurin} % Name 2. Betreuer
\newcommand{\betreuerIItaetigkeit}{Betriebliche Betreuerin} % Tätigkeit des Betreuers: Fachbetreuer, Wissenschaftlicher Betreuer, Betrieblicher Betreuer

\newcommand{\abschluss}{Master/Bachelor} % Art des Studiengangs/Abschlusses: Bachelor oder Master
\newcommand{\art}{Masterarbeit/Bachelorarbeit} % Art der Arbeit: Projektarbeit, Bachelorarbeit oder Masterarbeit

\newcommand{\zeitraum}{Januar 2025 -- Dezember 2025} % Monat JJJJ -- Monat JJJJ

\newcommand{\titel}{Titel der Arbeit} % Titel der Arbeit

\newcommand{\gebdatum}{01.\,01.\,2000} % Geburtsdatum, TT.\,MM.\,JJJJ

\newcommand{\gebort}{Leipzig} % Geburtsort

\newcommand{\datum}{01.\,12.\,2025} % z.B. Abgabedatum

% Fakultät und Studiengang:

\newcommand{\fak}{Musterwissenschaften} % Eingabe der Fakultät
\newcommand{\studiengang}{Musterstudiengang} % Eingabe des Studiengangs
 

\input{Konfigurationsdateien/Anhangsverzeichnis}


%Ebenen im Inhaltsverzeichnis
\newcommand{\nocontentsline}[3]{}
\newcommand{\tocless}[2]{\bgroup\let\addcontentsline=\nocontentsline#1{#2}\egroup}
\KOMAoptions{toc=indented}

%Verwendung normaler "Gänsefüßchen"
\MakeOuterQuote{"}

%Kein Einzug nach Absatz
\setlength\parindent{0pt}

%Schriftgröße in Tabellen gleich der Schriftgröße im Dokument
\usepackage{etoolbox}
\AtBeginEnvironment{table}{\sffamily}

%%%%%%%%%%%%%%%%%%%%%%%%%%%%%%%%%%%%%%%%%%%%%%%%%%%%%%%%%%%%%%%%%

%%%%%%%%%% Platz für weitere Pakete %%%%%%%%%%%%%%
%2 Bilder nebeneinander
\usepackage{subcaption}






%%%%%%%%%%%%%%%%%%%%%%%%%%%%%%%%%%%%%%%%%%%%%%%%%%

%-------------------------------------------------
%	Beginn des Dokuments und Einbinden der Kapitel
%-------------------------------------------------

\begin{document}

\begin{titlepage}
{\centering
{\Large \textbf{Hochschule für Technik, Wirtschaft und Kultur Leipzig}\par}
{\large \textbf{\fak} \par}
{\large \abschluss studiengang \stdgang\par}
\vspace{1.75cm}
{\Large \textbf{\titel}\par}
\vspace{1.5cm}
{\large \art \par}
\vspace{3cm}
{\large  von\par}
\vspace{1.25cm}
{\autor\\[3ex]
geb. am \gebdatum\\[3ex]
in \gebort\\[3ex]
\mnr\par}}
\vfill
{\noindent Verantwortlicher Hochschullehrer: \betreuerI 
\par \vspace{0.25cm}
\betreuerIItaetigkeit: \betreuerII
\par \vspace{0.75cm}
Leipzig, \zeitraum} % ggf. Ort anpassen
\end{titlepage}


%Aus dem Opal herunterladen, bspw. in einen Ordner "PDFs" hochladen, Einbinden durch einkommentieren (Prozentzeichen entfernen):

%\includepdf{PDFs/Sperrvermerk.pdf}

%Aufgabenstellung, bspw. in einen Ordner "PDFs" hochladen, Einbinden durch einkommentieren (Prozentzeichen entfernen):

%\includepdf{PDFs/Aufgabenstellung.pdf} 

\input{Konfigurationsdateien/Erklaerung}

% Danksagung, Einbinden der auzufüllenden Danksagung durch einkommentieren (Prozentzeichen entfernen):

%\input{Konfigurationsdateien/Danksagung}

\tableofcontents
\input{Konfigurationsdateien/Abbildungsverzeichnis}

%%%%%%Nutzung des Abkürzungs- und Formelzeichenverzeichnis%%%%%%%

%Für Nutzung des Abkürzungsverzeichnis folgendes einkommentieren (Prozentzeichen entfernen):

%%%%%%%%%%%%%%%%%%%%%%%%%%%%%%%%%%%
% Die Verwendung dieses Verzeichnisses ist optional!
%
% Es ist standardmäßig deaktiviert.
% Zum Aktivieren den Hinweis in der main.tex beachten.
%
% Weitere Informationen: https://www.namsu.de/Extra/pakete/Acronym.pdf
%
% Es muss immer mindestens ein Eintrag aus dem Verzeichnis im Text aufgerufen werden!
% ggf. mit Label-Prefixen arbeiten (z.B.: \abk{ak:ASTM})
%
\tocless\addchap{Abkürzungsverzeichnis}
%
%%%%%%%%%%Verzeichnis%%%%%%%%%%%%%
%
\begin{acronym}[LONGESTLONGEST]
	%\abb{Label}{Abkürzung}{Langfassung}
	\abk{astm}{ASTM}{American Society for Testing and Materials}
\end{acronym}
%
%%%%%%%%%%%%%%%%%%%%%%%%%%%%%%%%%%
%
% Aufruf des Kürzels im Text mit \acs{Label}
% Aufruf der Langfassung im Text mit \acl{Label}

%Für Nutzung des Formelverzeichnis folgendes einkommentieren (Prozentzeichen entfernen):

%%%%%%%%%%%%%%%%%%%%%%%%%%%%%%%%%%%
% Die Verwendung dieses Verzeichnisses ist optional!
%
% Es ist standardmäßig deaktiviert.
% Zum Aktivieren den Hinweis in der main.tex beachten.
%
% Es muss immer mindestens ein Eintrag aus dem Verzeichnis im Text aufgerufen werden!
%
\tocless\addchap{Formelverzeichnis}
%
%%%%%%%%%%Verzeichnis%%%%%%%%%%%%%
%
\begin{acronym}[LONGEST]
	%\form{Label}{Formelzeichen}{Einheit}{Beschreibung}
	\form{obfl}{A}{\si{\m^2}}{Oberfläche}
    \form{vol}{V}{\si{\m^3}}{Volumen}
\end{acronym}
%
%%%%%%%%%%%%%%%%%%%%%%%%%%%%%%%%%%
%
% Aufruf des Formelzeichens im Text mit \acs{Label}
% Aufruf der Beschreibung im Text mit \acl{Label}
% Aufruf von Beschreibung+Formelzeichen mit \acls{Label}


%%%%%%%%%%%%%%%%%%%%%%%%%%%%%%%%%%%%%%%%%%%%%%%%%%%%%%%%%%%%%%%%%

\input{Hauptkapitel/01Einleitung}
\input{Hauptkapitel/02Grundlagen}
\input{Hauptkapitel/03Material und Methoden}
\input{Hauptkapitel/04Ergebnisse}
\input{Hauptkapitel/05Auswertung}
\input{Hauptkapitel/06Zusammenfassung und Ausblick}


\begin{singlespacing}
\printbibliography
\end{singlespacing}

\appendix %Entfernen/Auskommentieren um Anhang zu deaktivieren
\input{Anhang} %Entfernen/Auskommentieren um Anhang zu deaktivieren

\end{document}
