%%%%%%%%%%%%%%%%%%%%%%%%%%%%%%%%%%
% Die Verwendung dieses Verzeichnisses ist optional!
%
% Es ist standardmäßig deaktiviert.
% Zum Aktivieren den Hinweis in der main.tex beachten.
%
% Es muss immer mindestens ein Eintrag aus dem Verzeichnis im Text aufgerufen werden!
% ggf. mit Label-Prefixen arbeiten (z.B.: \abb{fz:obfl})
%
% Konfiguration %

\newcommand{\form}[4]{\acro{#1}[\ensuremath{#2}]{#4{\acroextra{\hspace{5em}[#3]}}}}

\renewcommand*{\aclabelfont}[1]{\textbf{\boldmath\acsfont{#1}}}

\tocless\addchap{Formelverzeichnis}

\newcommand{\acls}[1]{\acl{#1} \acs{#1}}

%%%%%%%%%%Verzeichnis%%%%%%%%%%%%%
%

\begin{acronym}[LONGEST]
	%\form{Label}{Formelzeichen}{Einheit}{Beschreibung}
	\form{obfl}{A}{\si{\m^2}}{Oberfläche}
    \form{vol}{V}{\si{\m^3}}{Volumen}
\end{acronym}

%
%%%%%%%%%%%%%%%%%%%%%%%%%%%%%%%%%%


% Aufruf des Formelzeichens im Text mit \acs{Label}
% Aufruf der Beschreibung im Text mit \acl{Label}
% Aufruf von Beschreibung+Formelzeichen mit \acls{Label}
