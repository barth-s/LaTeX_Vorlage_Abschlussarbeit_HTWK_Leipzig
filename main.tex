%%%%%%%%%%%%%%%%%%%%%%%%%%%%%%%%%%
% HTWK Leipzig LaTeX Vorlage - Abschlussarbeit
% Interne Referenznummer 3_1 Overleaf
%
% Die Vorlage entspricht der "Richtlinie für die Anfertigung von wissenschaftlichen Arbeiten" der Fakultät Ingenieurwissenschaften
% Es besteht kein Anspruch auf Richtigkeit
% 
% Das Dokument basiert teilweise auf den Vorlagen von :
% Markus Voerkel
% https://de.overleaf.com/latex/templates/thesis-template-for-hochschule-fur-technik-wirtschaft-und-kultur-leipzig/nqpftcjmmtts
%
% Linda Vogel & Jon Arnt Kårstad
% https://www.overleaf.com/latex/templates/template-projekt-htwk/mphqwccfvvwy
%%%%%%%%%%%%%%%%%%%%%%%%%%%%%%%%%%
%
% Veröffentlicht unter CC BY-SA 4.0 - Lizenz
%
%---------------------------------------
%	Dokumentenklasse
%---------------------------------------

% Dokumentenklasse %
\documentclass[
ngerman,
toc=flat,
toc=chapterentrywithdots,
captions=tableabove,
listof=entryprefix,
listof=leveldown,
fontsize=12pt,
numbers=noenddot]
{scrreprt}

%---------------------------------------
%	Laden von Paketen
%---------------------------------------

\usepackage{babel}
\usepackage{lmodern}
\usepackage[T1]{fontenc}
\usepackage{float}
\usepackage{ragged2e}

% Geoemetry %
\usepackage{geometry}
\geometry{
	top = 25mm,
	headsep = 7mm,
	left = 28mm,
	right = 20mm, 
	bottom = 25mm,
	}

\usepackage[letterspace=150]{microtype}
\usepackage[onehalfspacing]{setspace}


% Caption %
\usepackage[labelfont={bf,sf},font={bf}, labelsep=space, singlelinecheck=off]{caption} 
\captionsetup[figure]{justification=centering}
\captionsetup[table]{justification=raggedright}


% Bibtex %
\usepackage[
    backend=biber,
    style=numeric-comp,
    sorting=none,
    defernumbers=true
]{biblatex}
\addbibresource{Literatur.bib}
\usepackage{csquotes}
\usepackage{amsmath}

\usepackage{graphicx}

% Hyperref %
\usepackage{hyperref}
\hypersetup{ 
    colorlinks=true,
    linkcolor=black,
    urlcolor=blue,
    citecolor=black}
 
% Einfügen von PDF-Dateien %
\usepackage{pdfpages}

% Einfacher Umgang mit Einheiten %
\usepackage{siunitx}
\sisetup{
    locale=DE,
    per-mode=fraction,
}

% Bessere Kompatibilität der Dokumentenklasse mit div. Paketen %
\usepackage{scrhack}

% Vorbereitung der Verzeichnisse %
\usepackage[printonlyused]{acronym}

%---------------------------------------
%	Weitere Konfigurationen
%---------------------------------------

%Kopfzeile mit Seitenzahl
\usepackage[autooneside=false]{scrlayer-scrpage}
\clearpairofpagestyles
\setkomafont{pageheadfoot}{\footnotesize}
\chead{{--~}\pagemark{~--}}


%Konfiguration Verzeichnisse
\BeforeStartingTOC[toc]{\singlespacing} 
\BeforeStartingTOC[lot]{\renewcommand\autodot{:}}
\BeforeStartingTOC[lof]{\renewcommand\autodot{:}}


%Anpassung der Kapitelüberschrift
\renewcommand*{\chapterpagestyle}{scrheadings}
\RedeclareSectionCommand[%
beforeskip=0pt,
afterskip=16pt,
afterindent = false,
font=\LARGE]{chapter}

%Verhindere Zeilenumbruch innerhalb \cite[Seite]{quelle}
\renewcommand*{\prenotedelim}{\addnbspace}
\renewcommand*{\postnotedelim}{\addcomma\addnbspace}
\renewcommand*{\multicitedelim}{\addcomma\addnbspace}
\renewcommand*{\extpostnotedelim}{\addnbspace}
\renewcommand*{\volcitedelim}{\addcomma\addnbspace}


%%%%%%%%%%%%%%%%%%%%%%%%%%%%%%%%%%%%%%%%%%%%%%%%%
%
%       Bitte hier die eigenen Daten eingeben
%       (für Generierung der Titelseite etc.)
%
%%%%%%%%%%%%%%%%%%%%%%%%%%%%%%%%%%%%%%%%%%%%%%%%%

\newcommand{\autor}{Max Mustermann} % Vorname Name
\newcommand{\mnr}{12345} % Matrikelnummer, bspw.: 12345

\newcommand{\betreuerI}{Prof. Dr.-Ing. Moritz Musterprof} % Name 1. Betreuer bzw. betreuender Prof

\newcommand{\betreuerII}{Dipl.-Ing. Manuela Musteringenieurin} % Name 2. Betreuer
\newcommand{\betreuerIItaetigkeit}{Betriebliche Betreuerin} % Tätigkeit des Betreuers: Fachbetreuer, Wissenschaftlicher Betreuer, Betrieblicher Betreuer

\newcommand{\abschluss}{Master/Bachelor} % Art des Studiengangs/Abschlusses: Bachelor oder Master
\newcommand{\art}{Masterarbeit/Bachelorarbeit} % Art der Arbeit: Projektarbeit, Bachelorarbeit oder Masterarbeit

\newcommand{\zeitraum}{Januar 2025 -- Dezember 2025} % Monat (JJJJ) -- Monat JJJJ

\newcommand{\titel}{Titel der Arbeit} % Titel der Arbeit

\newcommand{\gebdatum}{01.\,01.\,2000} % Geburtsdatum, TT.\,MM.\,JJJJ

\newcommand{\gebort}{Leipzig} % Geburtsort

\newcommand{\datum}{01.\,12.\,2025} % z.B. Abgabedatum

% Fakultäten, nicht ändern!
\newcommand{\FAS}{Architektur und Sozialwissenschaften}
\newcommand{\FB}{Bauwesen}
\newcommand{\FING}{Ingenieurwissenschaften}
\newcommand{\FDIT}{Digitale Transformation}
\newcommand{\FIM}{Informatik und Medien}
\newcommand{\FWW}{Wirtschaftswissenschaft und Wirtschaftsingenieurwesen}

% Auswahl Fakultät und Studiengang:
\newcommand{\fak}{\FING} % Eingabe der Fakultät (entsprechend Kürzel oben)
\newcommand{\studiengang}{Musteringenieurwesen} % bspw.: Maschinenbau 

%%%%%%%%%%%%%%%%%%%%%%%%%%%%%%%%%%
% Konfiguration des Anhangsverzeichnis
%
% weitere Informationen: https://komascript.de/comment/5578#comment-5578, mit Anpassungen
% Beispiel: https://komascript.de/comment/5609#comment-5609
%
%%%%%%%%%%%%%%%%%%%%%%%%%%%%%%%%%%

\DeclareNewTOC[%
  owner=\jobname,
  listname={Anhangsverzeichnis},% Titel des Verzeichnisses
]{atoc}% Dateierweiterung (a=appendix, toc=table of contents)
\DeclareNewTOC[%
  listname={Abbildungen im Anhang},% Titel des Verzeichnisses
  name=\noexpand\listoflofentryname,
]{alof}% Dateierweiterung (a=appendix, lof=list of figures)
\DeclareNewTOC[%
  listname={Tabellen im Anhang},% Titel des Verzeichnisses
  name=\noexpand\listoflotentryname
]{alot}% Dateierweiterung (a=appendix, lot=list of tables)

\makeatletter
\newcommand*{\useappendixtocs}{%
  \renewcommand*{\ext@toc}{atoc}%
  \scr@ifundefinedorrelax{hypersetup}{}{% damit es auch ohne hyperref funktioniert
    \hypersetup{bookmarkstype=atoc}%
  }%
  \renewcommand*{\ext@figure}{alof}%
  \renewcommand*{\ext@table}{alot}%
}
\newcommand*{\usestandardtocs}{%
  \renewcommand*{\ext@toc}{toc}%
  \scr@ifundefinedorrelax{hypersetup}{}{% damit es auch ohne hyperref funktioniert
    \hypersetup{bookmarkstype=toc}%
  }%
  \renewcommand*{\ext@figure}{lof}%
  \renewcommand*{\ext@table}{lot}%
}
\scr@ifundefinedorrelax{ext@toc}{%
  \newcommand*{\ext@toc}{toc}
  \renewcommand{\addtocentrydefault}[3]{%
    \expandafter\tocbasic@addxcontentsline\expandafter{\ext@toc}{#1}{#2}{#3}%
  }
}{}
\makeatother
 
\usepackage{xpatch}
\xapptocmd\appendix{%
  \addpart{\appendixname}
  \useappendixtocs
  \listofatocs

}{}{}

\BeforeStartingTOC[atoc]{\singlespacing} 
\BeforeStartingTOC[alot]{\singlespacing\renewcommand\autodot{:}}
\BeforeStartingTOC[alof]{\singlespacing\renewcommand\autodot{:}}%



%Ebenen im Inhaltsverzeichnis
\newcommand{\nocontentsline}[3]{}
\newcommand{\tocless}[2]{\bgroup\let\addcontentsline=\nocontentsline#1{#2}\egroup}

%Verwendung normaler "Gänsefüßchen"
\MakeOuterQuote{"}

%Kein Einzug nach Absatz
\setlength\parindent{0pt}

%Schriftgröße in Tabellen gleich der Schriftgröße im Dokument
\usepackage{etoolbox}
\AtBeginEnvironment{table}{\sffamily}

%%%%%%%%%%%%%%%%%%%%%%%%%%%%%%%%%%%%%%%%%%%%%%%%%%%%%%%%%%%%%%%%%


%%%%%%%%%% Platz für weitere Pakete %%%%%%%%%%%%%%
%2 Bilder nebeneinander
\usepackage{subcaption}






%%%%%%%%%%%%%%%%%%%%%%%%%%%%%%%%%%%%%%%%%%%%%%%%%%

%-------------------------------------------------
%	Beginn des Dokuments und Einbinden der Kapitel
%-------------------------------------------------

\begin{document}

\begin{titlepage}
{\centering
{\Large \textbf{Hochschule für Technik, Wirtschaft und Kultur Leipzig}\par}
{\large \textbf{Fakultät \fak} \par}
{\large \abschluss-Studiengang \studiengang\par}
\vspace{1.75cm}
{\Large \textbf{\titel}\par}
\vspace{1.25cm}
{\large \abschluss arbeit Nr. \arbeitsnr\par}
\vspace{2.5cm}
{\large  von\par}
\vspace{1.5cm}
{\autor\\[3ex]
geb. am \gebdatum\\[3ex]
in \gebort\\[3ex]
\mnr\par}}
\vfill
{\noindent Verantwortlicher Hochschullehrer: \betreuerI

\vspace{0.25cm}
Betrieblicher Betreuer: \betreuerII

\vspace{1cm}
Leipzig, \zeitraum} %Ort ggf. Anpassen
\end{titlepage}

%Aus dem Opal herunterladen, Einbinden durch einkommentieren
%\includepdf{Inhalte/PDFs/Sperrvermerk.pdf}

%Aufgabenstellung, Einbinden durch einkommentieren:
%\includepdf{Inhalte/PDFs/Aufgabenstellung.pdf} 

\thispagestyle{empty}
\begin{center}
\large \lsstyle Erklärung
\end{center}
\vspace{1.5cm}
{\doublespacing
Ich versichere wahrheitsgemäß, die \art{} selbständig angefertigt, alle benutzten Hilfsmittel vollständig und genau angegeben und alles kenntlich gemacht zu haben, was aus Arbeiten anderer unverändert oder mit Abänderungen entnommen wurde.}\par
\vspace{2cm}
\noindent
%Autor
\begin{minipage}[t]{6.5cm}
% gepunktete Linie
\dotfill \par
% Text unter der Linie
\onehalfspacing
\autor \par
Leipzig, den \datum %ggf. Ort  Anpassen
\end{minipage}\par
\vspace{2cm}
\clearpage

%Danksagung, Einbinden der auzufüllenden Danksagung.tex durch einkommentieren:
%\thispagestyle{empty}
\begin{center}
\large \lsstyle
Danksagung
\end{center}
\vspace{1.5cm}
Ich bedanke mich bei meinem Betreuer \betreuerI{}. Auch \betreuerII{} möchte ich für ihre Unterstützung danken...

\tableofcontents

\tocless\addchap{Abbildungs- und Tabellenverzeichnis}
\listoffigures
\pagebreak %Nach Belieben entfernen
\listoftables

%%%%%%Nutzung des Abkürzungs- und Formelzeichenverzeichnis%%%%%%%

%Für Nutzung des Abkürzungsverzeichnis folgendes auskommentieren (Prozentzeichen entfernen):

%%%%%%%%%%%%%%%%%%%%%%%%%%%%%%%%%%%
% Die Verwendung dieses Verzeichnisses ist optional!
%
% Es ist standardmäßig deaktiviert.
% Zum Aktivieren den Hinweis in der main.tex beachten.
%
% Weitere Informationen: https://www.namsu.de/Extra/pakete/Acronym.pdf
%
% Es muss immer mindestens ein Eintrag aus dem Verzeichnis im Text aufgerufen werden!
%
\tocless\addchap{Abkürzungsverzeichnis}
%
%%%%%%%%%%Verzeichnis%%%%%%%%%%%%%
%
\begin{acronym}[LONGESTLONGEST]
	%\abb{Label}{Abkürzung}{Langfassung}
	\abb{astm}{ASTM}{American Society for Testing and Materials}
\end{acronym}
%
%%%%%%%%%%%%%%%%%%%%%%%%%%%%%%%%%%
%
% Aufruf des Kürzels im Text mit \acs{Label}
% Aufruf der Langfassung im Text mit \acl{Label}


%Für Nutzung des Formelverzeichnis folgendes auskommentieren (Prozentzeichen entfernen):

%%%%%%%%%%%%%%%%%%%%%%%%%%%%%%%%%%%
% Die Verwendung dieses Verzeichnisses ist optional!
%
% Es ist standardmäßig deaktiviert.
% Zum Aktivieren den Hinweis in der main.tex beachten.
%
% Es muss immer mindestens ein Eintrag aus dem Verzeichnis im Text aufgerufen werden!
%
\tocless\addchap{Formelverzeichnis}
%
%%%%%%%%%%Verzeichnis%%%%%%%%%%%%%
%
\begin{acronym}[LONGEST]
	%\form{Label}{Formelzeichen}{Einheit}{Beschreibung}
	\form{obfl}{A}{\si{\m^2}}{Oberfläche}
    \form{vol}{V}{\si{\m^3}}{Volumen}
\end{acronym}
%
%%%%%%%%%%%%%%%%%%%%%%%%%%%%%%%%%%
%
% Aufruf des Formelzeichens im Text mit \acs{Label}
% Aufruf der Beschreibung im Text mit \acl{Label}
% Aufruf von Beschreibung+Formelzeichen mit \acls{Label}


%%%%%%%%%%%%%%%%%%%%%%%%%%%%%%%%%%%%%%%%%%%%%%%%%%%%%%%%%%%%%%%%%

\chapter{Einleitung}
\chapter{Grundlagen}
\chapter{Material und Methoden}
\chapter{Ergebnisse}
\chapter{Auswertung}
\chapter{Zusammenfassung und Ausblick}


\begin{singlespacing}
\printbibliography
\end{singlespacing}

\appendix %Entfernen/Auskommentieren um Anhang zu deaktivieren
\chapter{Beispiel 1} %Entfernen/Auskommentieren um Anhang zu deaktivieren

\end{document}
